\documentclass{amsart}
\usepackage{amsmath,amssymb,amsthm}

\newtheorem{theorem}{Theorem}[section]
\newtheorem{conjecture}[theorem]{Conjecture}

\theoremstyle{definition}
\newtheorem{definition}[theorem]{Definition}
\newtheorem{problem}[theorem]{Problem}

\theoremstyle{remark}
\newtheorem*{remark}{Remark}

\title{The Problem of the Randomly Walked Dog}
\author{Scallop Ye}

\begin{document}

\begin{abstract}
    In 1905, Karl Pearson \cite{pearson} proposed on \emph{Nature} the
    famous problem of the random walk. This article presents a
    new problem analogous to Pearson's one and formulates several
    conjectures based on computer simulation.
\end{abstract}

\maketitle

\section{Introduction}

A man each day takes his dog out for a walk. For training purposes,
the dog is required to come up with a list of steps as (direction, distance) pairs
in which no two directions are identical or opposite and every distance is nonzero.
The man then shuffles the steps uniformly at random and begins to walk the dog.

They start from a point $O$, turn to the direction of the first step and walk the
corresponding distance in a straight line, leaving a trail behind them.
They repeat this process with the next step until no more step is available.
The dog wins a treat if they ever crossed the trail before they reach the destination.

Now, the dog asks you to help maximize the probability to win the treat in a walk.
The man is known to sometimes restrict the number of steps provided and sometimes not.
Note that too much calculation exhausts the dog.

\begin{remark}
    As one might have seen, this problem differs from Pearson's one in that
    it is not fundamentally probabilistic. We could also present the problem
    without introducing randomness, except that it would make the storytelling less convenient.
\end{remark}

\section{Preliminaries}

\begin{definition}
    Let $P=(p_k)_{k=1}^n$ denote a \emph{polygonal path} with vertices\\
    $(p_1,p_2,\dots,p_n)$ and edges
    $(\overline{p_1p_2},\overline{p_2p_3},\dots,\overline{p_{n-1}p_n})$.
    The set of all the points on the path $P$, denoted by $\widetilde{P}$,
    is the union of all its edges:
    \[\widetilde{P}=\bigcup_{i=1}^{n-1}\overline{p_{i}p_{i+1}}.\]
\end{definition}

\begin{definition}
    A polygonal path $(p_k)_{k=1}^n$ is \emph{simple} if and only if
    for all $i,j$ such that $1<i\le j<n$,
    \[
        \overline{p_{i-1}p_i}\cap\overline{p_jp_{j+1}}=
        \begin{cases}
            \{p_i\}            & \text{if $i=j$},                     \\
            \{p_1\}\cap\{p_n\} & \text{if $n>3$ and $(i,j)=(2,n-1)$}, \\
            \varnothing        & \text{otherwise}.
        \end{cases}
    \]
\end{definition}

\newcommand{\sseq}{\widetilde{\Delta}_n}
\newcommand{\sset}{\Delta_n}

\begin{definition}
    A \emph{step sequence} of length $n$, denoted by $\sseq=(\delta_k)_{k=1}^n$,
    is a sequence of distinct \textbf{noncollinear} vectors in $\mathbb{R}^2$.
    The set of elements in $\sseq$ is denoted by $\sset$.
    We call $\sset$ a \emph{step set} of size $n$ and $\sseq$
    a \emph{permutation} of $\sset$.

    The \emph{resultant walk} of $\sseq$, denoted by $\mathrm{walk}(\sseq)$, is the polygonal path
    $(p_k)_{k=0}^{n}$ where $p_0=(0,0)$ and \[p_i=p_{i-1}+\delta_i,\quad1\le i\le n.\]
\end{definition}

\begin{theorem}
    For all step sets $\sset$, there exists a permutation $\sseq$ of $\sset$
    such that $\mathrm{walk}(\sseq)$ is simple.
\end{theorem}

The proof of this theorem is left as an exercise to the reader.

\newcommand{\csset}{C_n}
\newcommand{\csseq}{\widetilde{C}_n}

\begin{problem}
\label{problem:general}
Define the \emph{simplicity} of a step set as the number of its permutations
of which the resultant walk is simple. For a fixed $n\ge2$, denote by $\csset$
a step set with the minimum simplicity among all $\sset$.
We therefore call $\csset$ a \emph{complex} step set. Let $a_n$ be the
simplicity of $\csset$. Find an instance of $\csset$ and the value of $a_n$.
\end{problem}

\begin{problem}
\label{problem:zerosum}
For a fixed $n\ge3$, let $b_n$ be the minimum simplicity of a step set of size $n$
in which all the vectors \textbf{sum to zero}. Find the value of $b_n$.
\end{problem}

\section{Results}

After computer simulation on Problems \ref{problem:general} and \ref{problem:zerosum},
we formulate the following conjectures from our observations.

\begin{conjecture}[Initial terms of and relation between $(a_n),(b_n)$]
    \begin{align*}
        (a_2,a_3,\dots,a_6) & =(2,4,8,28,100),                   \\
        (b_3,b_4,\dots,b_7) & =(6,16,40,168,700),                \\
        b_n                 & =n\cdot a_{n-1}\quad\forall n\ge3.
    \end{align*}
\end{conjecture}

\begin{conjecture}[Law of large complex step sets]
    The sequence \[\left(\frac{a_n}{n!}\right)_{n=2}^\infty\]
    is strictly decreasing and converges to zero.
\end{conjecture}

\begin{conjecture}[Property of complex step sets]
    Let $\csset$ be a complex step set of size $n$. For all permutations
    $\csseq$ of $\csset$ with
    \[P=\mathrm{walk}(\csseq)=(p_k)_{k=0}^{n},\]
    it holds that $p_0\ne p_n$ and
    \[\widetilde{P}\cap\overline{p_0p_n}=\{p_0,p_n\}.\]
\end{conjecture}

Although we are not yet able to prove these conjectures or to
find a general solution to the problem, we can still make
some practical suggestions for the dog:

\begin{enumerate}
    \item The more steps you provide, the better.
    \item If the number of steps is small, it is possible to brute-force
          an optimal solution by random simulation.
    \item If the number of steps is very large, a list of fixed-distance steps with uniformly
          distributed directions might serve to win the treat almost surely.
    \item No matter how hard you try, it is always possible that you fail to win the treat.
          Keep trying, don't lose faith, and don't exhaust yourself.
\end{enumerate}

\begin{thebibliography}{9}

    \bibitem{pearson}
    Pearson, K.
    The Problem of the Random Walk.
    \emph{Nature}
    \textbf{72},
    294 (1905).

\end{thebibliography}

\end{document}